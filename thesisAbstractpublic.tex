% Public thesis abstract goes here
% The goal of the public abstract is to explain how the research advances knowledge and addresses solutions to problems facing society.
In this work, multiple neural network techniques are presented for use in educational measurement applications. In educational measurement, the goal is to quantify student learning based on assessment performance. Item Response Theory (IRT) models the probability of students answering individual questions correctly as a function of the particular student's set of latent abilities (e.g. add, subtract, multiply, divide), as well as parameters pertaining to the specific question (e.g. difficulty).

A common task is to measure a population of students' knowledge, given their responses to an assessment. Though many methods exist for this purpose, measurement becomes difficult or infeasible as the number of students, questions, and latent abilities increase. This issue has become more problematic in the age of Big Data, as electronic learning environments increase in functionality and popularity and educational assessments grow in scale and scope.

In this thesis, a novel neural network architecture is presented which estimates question and student parameters from high-dimensional assessment data with ease. The modifications made to existing deep learning methods provide a direct link to IRT and have clear use in applications such as AI tutoring systems. The proposed machine learning architecture is relevant from perspectives outside of education as well, as it introduces interpretability into otherwise black-box deep learning models.

