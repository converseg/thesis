% Public thesis abstract goes here
In this work, multiple neural network techniques are presented for use in educational measurement applications. In educational measurement, the goal is to quantify the learning of students, based on their performance on assessments. Item Response Theory (IRT) models the probabilty of students answering individual questions correctly as a function of the particular student's set of latent skills (e.g. add, subtract, multiply, divide), as well as parameters pertaining to the specific question, such as difficulty.

A common task is to measure a population of students' latent knowledge from their assessment responses. Though many methods exist for this purpose, measurement becomes difficult as the number of students, items, and latent abilities increase. Specifically, when there are more than ten latent traits present, inferring student knowledge becomes infeasible. In this thesis, a novel neural network architecture is presented which estimates item and student parameters from high-dimensional assessment data with ease. This architecture is interesting from perspectives outside of education as well, as it introduces interpretability into an otherwise black-box machine learning model.

Knowledge tracing is an application which has gained more attention recently, due to the emergence of online learning and AI tutoring systems. In an online environment where many items are available, the goal is to track student's knowledge dynamically as they progress through an assessment. This information can be used by intelligent tutoring systems to select which items to present to students, based on their personalized needs. In this thesis, modifications from the parameter estimation application are extended in order to integrate IRT into deep knowledge tracing models. The link between IRT and knowledge tracing provides additional model explainability, while remaining competitive with state-of-the-art methods.

