% Public thesis abstract goes here
% The goal of the public abstract is to explain how the research advances knowledge and addresses solutions to problems facing society.
In this work, multiple neural network techniques are presented for use in educational measurement applications. In educational measurement, the goal is to quantify the learning of students, based on their performance on assessments. Item Response Theory (IRT) models the probability of students answering individual questions correctly as a function of the particular student's set of latent skills (e.g. add, subtract, multiply, divide), as well as parameters pertaining to the specific question, such as difficulty.

A common task is to measure a population of students' latent knowledge, given their responses to an assessment. Though many methods exist for this purpose, measurement becomes difficult or infeasible as the number of students, questions, and latent abilities increase. This issue has become more problematic in the age of Big Data, as electronic learning environments increase in functionality and popularity and educational assessments grow in scale and scope.

In this thesis, a novel neural network architecture is presented which estimates question and student parameters from high-dimensional assessment data with ease. The modifications made to existing deep learning methods provide a direct link to IRT and have clear use in applications such as AI tutoring systems. The proposed machine learning architecture is interesting from perspectives outside of education as well, as it introduces interpretability into otherwise black-box deep learning models.

%Due to the more recent emergence of online learning and AI tutoring systems, the application of knowledge tracing has gained more attention. In an online environment where many items are available, the goal is to track student's learning dynamically as they progress through an assessment. This information can be used by intelligent tutoring systems to select which items to present to students, based on their personalized needs. In this thesis, modifications from the parameter estimation application are extended in order to integrate IRT into deep knowledge tracing models. The link between IRT and knowledge tracing provides additional model explainability, while remaining competitive with state-of-the-art methods.

%The main contribution of this thesis is to present alternative methods for measuring student learning which thrive the the age of big data. This is helpful in the application area of education research and measurement -- where traditional techniques face computational difficulties, the proposed machine learning driven approaches can efficiently handle high-dimensional data. From the computer science perspective, interpretability is inserted into otherwise unexplainable deep neural networks using domain-specific knowledge.
