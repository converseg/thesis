\chapter{Introduction}

In educational measurement, the goal is to quantify student learning in order to better assess the needs of individuals. In particular, Item Response Theory (IRT) provides a means of modeling the probability of students answering items (i.e. questions) of an assessment correctly \cite{lord1968}, dependent on both the student's latent ability (i.e. knowledge, skill) and the difficulty and relevance of the item. IRT quantifies a student's latent ability as a continuous value $\vect \Theta$. In this work, we focus on the multidimensional case where $\vect \Theta$ is a vector representing multiple skills required for a particular assessment \cite{reckase2009multidimensional}, such as the skills ``add,'' ``subtract,'' ``multiply,'' and ``divide'' on an elementary math exam.

In real applications, the latent trait $\vect \Theta$ is unobservable, and the task is to infer each student's knowledge level from their correct and incorrect responses on an assessment. Many methods exist for this task, focusing around maximum likelihood estimation \cite{baker_kim2004, chen2019}, the EM algorithm \cite{bock1981, Feng2014}, and Monte-Carlo methods \cite{cai2010, patz1999}. These are all popular parameter estimation techniques in computational statistics, yet experience difficulties when faced with high-dimensional large datasets. Specifically, when the number of students, items, and latent abilities increase, these traditional approaches become infeasible due to computational complexity. 

Recent years have seen an upswing in the popularity of artificial neural networks in many fields, due to their strong predictive power and ability to learn deep contextual representations \cite{neural_net, vaswani2017}. But this high level of accuracy comes with the price of explainability -- most neural network models act as a black-box, where the decision-making process and understanding of parameters is not interpretable \cite{zhang2018interpretable}.

The primary contribution of this thesis is the development of an auto-encoding neural network approach to IRT parameter estimation which remains efficient on large datasets (a sizable hurdle of traditional IRT methods), while introducing explainability into the neural network (a drawback of deep learning models). The proposed method, titled ``ML2P-VAE,'' modifies the architecture of a variational autoencoder (VAE) \cite{kingma2014} using domain knowledge of experts in the educational field \cite{daSilva2018, guo2017} in order to convert the VAE decoder into an approximate IRT model, parameterized and learned by a neural network. Despite being an unsupervised learning method, ML2P-VAE allows for interpretation of a hidden neural layer as estimates to student's latent ability $\vect \Theta$.

The significance of the novel ML2P-VAE method is summarized as follows. The use of a VAE, rather than a regular autoencoder \cite{guo2017}, greatly increases the accuracy of parameter estimates. In comparison to traditional IRT estimation methods, ML2P-VAE achieves competitive performance in terms of accuracy, while providing a speedup in computation time. On high-dimensional datasets, the proposed method yields accurate estimates within reasonable runtime, while traditional methods are unable to produce any result. Stepping away from educational measurement, the architecture of ML2P-VAE provides a high level of interpretability to a neural network model. Additionally, a new VAE architecture is developed which allows for correlated latent code -- an idea which can be extended to other fields where prior knowledge of the abstract latent space is available.

Technology has also had an impact on education which is more visible on the user side -- electronic learning environments have been around for a while, but are increasing in popularity and adding more features. Such features include computerized adaptive testing and intelligent tutoring systems, which can tailor content and give feedback to users on a more personal level based on their particular needs \cite{meijer1999, ong2003}.

These applications require the ability to evaluate student learning in real-time, updating the estimate of student knowledge $\vect \Theta$ both as more data becomes available and as students learn from their previous experiences. This is the task of knowledge tracing, originally introduced via a Bayesian approach in 1995 \cite{corbett1995} to track the progress of students learning to program. More recently, deep neural networks have become the standard in knowledge tracing, including the use of LSTMs \cite{piech2015} and attention-based neural networks \cite{pandey2019}.

Deep knowledge tracing methods take a partial sequence of student interactions, each of which includes an indicator of the question answered, and whether the item was answered correctly or not. The output is a predicted probability that the student will answer the \textit{next} item in the sequence correctly. While state-of-the-art models are able to approximate this probability with high precision \cite{zhang2017}, it is the only useful measure of student knowledge provided by the model which is readily available, though other post hoc techniques can be applied. This points back towards the previously discussed black-box nature of deep neural networks.

A contribution of this thesis is a modification to existing knowledge tracing methods which links the knowledge tracing framework directly to Item Response Theory. Since the output of all knowledge tracing models is the probability of success on the next interaction, it makes sense to utilize the theory of IRT latent trait models in this application. Using a similar technique as in Part I, the expert-annotated item-skill association is incorporated into the neural architecture, allowing for interpretation of a hidden layer as estimates to a student's latent ability. This explicit representation of $\vect \Theta$ is very convenient in practice, and allows knowledge tracing models to also act as parameter estimation methods. This presents a trade-off between interpretability and accuracy, though the proposed ``IRT-inspired knowledge tracing'' methods still remain competitive with other models.

\section{Organization}
This thesis is organized into two parts. In Part I, Chapters 2-4 introduce Item Response Theory (IRT) and analyze the novel parameter estimation method, ML2P-VAE. This method uses a modified variational autoencoder to estimation parameters in IRT models. Chapter 2 provides background on two relevant neural network models (autoencoders and variational autoencoders), along with a summary of traditional IRT parameter estimation methods. Chapter 3 describes the ML2P-VAE method in detail, including a software package which was developed for easy implementation of the method. Chapter 4 summarizes all results and experiments performed on educational datasets with ML2P-VAE.

In Part II, Chapters 5-7 explore knowledge tracing, a task commonly seen in online learning environments. While other deep learning methods for knowledge tracing lack interpretability, new methods presented in this thesis sacrifice prediction power for explainability. Chapter 5 gives background on time-dependent neural networks such as RNN, LSTM, and Transformers, along with a literature review of other popular knowledge tracing methods. Chapter 6 proposes a method of incorporating Item Response Theory models into the knowledge tracing framework. Chapter 7 presents results of the interpretable adaptation to knowledge tracing, including comparisons to other methods in the literature.

In the final chapter of this thesis, Chapter 8 briefly explores some applications outside of education which the ML2P-VAE method can be applied to. These areas include the health sciences, where IRT models have previously been applied to the Beck Depression Inventory \cite{beck1996, fragoso2013, huang2015} and personality questionnaires such as the Big Five \cite{robie2001}. A new extension to sports analytics is explored as well, using ML2P-VAE to evaluate the skill of professional baseball players. Finally, the contributions and significance of this thesis work as a whole is summarized in the conclusion.

