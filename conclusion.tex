\chapter{Discussion}\label{ch:conclusion}

\section{Applications Outside of Education}\label{sec:related_work}
The ML2P-VAE method, introduced in Chapter \ref{ch:ml2pvae_methods} can be applied to areas outside of the realm of education. In general, it can be useful when some unobservable latent code is assumed to generate the observable data, and domain knowledge is available about said latent code. The ML2P-VAE method can be of use in most areas where other confirmatory IRT models are commonly implemented. 

\subsection{Health Sciences Questionnaires}
BDI, personality 

\subsection{Sports Analytics}
\sideremark{TODO: edit}
In this section, we summarize the publication ``Variational Autoencoders for Baseball Player Evaluation,'' \cite{fsdm_paper} presented at the International Conference on Fuzzy Systems and Data Mining, 2019. In this work, the idea of latent traits influencing observable performance is extended from educational measurement to professional sports, specifically player evaluation. Though there does not exist any formal theory relating to latent athletic skills such as what education has in IRT, parallels can still be drawn between the two fields.

This work focuses on professional baseball players. In education, the ``observable performance'' was given by binary responses to questions on exams -- here, the observable data is recorded statistics from a player in a particular season, such as hits, stolen bases, home runs, or strikeouts. Where IRT focused on estimating student's latent knowledge $\vect \Theta$, the goal of this work is to develop \textit{new} measures to quantify a baseball player's skills on the field, which influence what is observed in the box score.

The ideas of the ML2P-VAE model can be used for this task. We first define four underlying skills required for baseball players to be successful: \textit{contact} (how often does the batter hit the ball), \textit{power} (how hard does the player hit the ball), \textit{baserunning} (is the player good at running the bases), and \textit{pitch intuition} (does the player swing when they are supposed to). In ML2P-VAE, the item-skill relationship given by the $Q$-matrix is instrumental in allowing interpretation of a hidden neural network layer. A similar binary matrix is developed in the baseball application, relating the four skills to the observed data. For example, the statistic ``home runs'' requires only the power skills, while avoiding ``strikeouts'' requires both contact and pitch intuition.

After training the modified VAE, we can obtain quantities for every baseball player in each of the four skills. In order to evaluate the effectiveness of our skill values, we compare the skills outputted by the VAE encoder to the commonly used baseball statistics such as contact rate, speed score, isolated power, and on-base percentage \cite{baseball_reference}. We don't strive to exactly replicate these statistics (since we are developing \textit{new} measures), but it can be helpful to reference in order to confirm that our measures actually represent what we intend them to quantify. For example, the players with the largest \textit{power} value include Barry Bonds and Mark McGwire (notable power hitters), and the players with the highest \textit{baserunning} value are Rickey Henderson and Lou Brock (famous base-stealers).

Though there is less mathematical theory supporting the application of ML2P-VAE to the sports analytics field, it still yields interesting results. We are able to interpret a hidden layer of the neural network, and use those values to quantify the latent code's influence on observable performance. The proposed measures of underlying athletic skills can be useful for sports franchises in evaluating a player's usefulness to a team.

\section{Conclusion}
\sideremark{TODO -- kind of repeat the introduction, but with notation and terminology. Focus more on the results and implications of the research}

In this thesis, novel neural network methods for application in educational measurement are presented and described in detail. From a perspective outside of the application area, the work is interesting from a machine learning point of view, as modifications to neural architecture are made which allow for direct interpretation of trainable parameters and a hidden layer.

The primary research described in Chapter \ref{ch:ml2pvae_methods} details the ML2P-VAE method for IRT parameter estimation. Using domain-specific knowledge of the $Q$-matrix detailing the binary relationship between items and skills, the weights matrix of the VAE decoder is constrained. This causes the generative model $p_\beta(\vect x | \vect z)$ from Section \ref{sec:vae_derive} matches the exact form of the ML2P model in Equation \ref{eq:ml2pq}. After training ML2P-VAE, the VAE decoder becomes an approximate ML2P model, and the encoder maps a student's binary responses to an estimate of the latent ability $\vect \Theta$. At the very least, the ML2P-VAE method is an interesting and unorthodox use of a variational autoencoder which yields comparable accuracy to that of more traditional IRT methods. But additionally, it holds a considerable practical advantage over these other methods in the context of big data and computational complexity.

The ML2P-VAE technique avoids the curse of dimensionality, a significant problem faced by traditional IRT parameter estimation methods explored in Section \ref{sec:dim}. Since most other methods rely on numerical integration or MCMC methods, estimating parameters where the dimension of the latent traits $\vect \Theta \in\R^K$ is larger than $K=10$ is difficult, and quickly becomes unobtainable as $K$ increases. Rather than computing a $K$-dimensional integral, ML2P-VAE uses a neural network to learn a $K$-dimensional posterior distribution. In ANN, the scale of $K <<< 1000$ \sideremark{TODO: write this better} does not present any computational difficulty.

In addition to scalability within the area of educational measurement, the ML2P-VAE method provides a novel VAE architecture which allows for correlated latent code. Though most VAE implementations assume that $\vect z \sim \mathcal{N}(0,I)$, if domain knowledge about the latent space $\vect z \in \R^K$ is available, then the proposed neural architecture can fit a VAE to the more general multivariate Gaussian distribution $\mathcal{N}(\vect \mu, \Sigma)$. As discussed in Section \ref{sec:cov}, this is non-trivial and we prove that for any input $\vect x_0 \in \R^n$, the proposed VAE encoder will output a posterior probability distribution $q_\alpha(\vect z | \vect x_0) = \mathcal{N}(\vect \mu_0, \Sigma_0)$. The VAE design presented in this thesis is of interest in other fields which wish to impose additional structure on the latent code.

% online learning and knowledge tracing
A second research project extends the ideas from ML2P-VAE to the knowledge tracing problem, where the learning environment is much more dynamic. While deep knowledge tracing models are capable of predicting student success on their next interaction, this computation is not explainable. The work in this thesis suggests incorporating Item Response Theory into the knowledge tracing framework in order to provide an explicit representation of student knowledge $\vect \Theta$ over time.

The IRT-inspired knowledge tracing methods can be integrated directly into many other knowledge tracing models, including DKT \cite{piech2015} and SAKT \cite{pandey2019}. Similar to ML2P-VAE, the core idea is to inject domain knowledge into the neural architecture to obtain an approximate IRT model. This presents a trade-off between prediction power and interpretability, while remaining competitive with state-of-the-art models in terms of test AUC scores.

% wrap it all up.
In both research areas, there is room for future work. ML2P-VAE can be extended to other IRT models to include a guessing parameter or a graded response model, discussed in Section \ref{sec:ml2pvae_future}. The notion of incorporating expert annotation into VAE architecture can be applied to areas in the health sciences, described in Section \ref{sec:related_work}. Section \ref{sec:kt_future} describes additional paths to inject IRT into the framework of deep knowledge tracing. The use of deep learning methods in education is becoming more popular, and the work in this thesis explores a few of many avenues for potential research. The contributions detailed here stand on their own in both the studies of machine learning and educational measurement, and identify interesting parallels between the two fields.

