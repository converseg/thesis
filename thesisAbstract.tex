%2-page limit
% Thesis abstract goes here

In educational measurement, Item Response Theory (IRT) provides a means of quantifying student knowledge. Specifically, IRT models the probability of a student answering a particular item correctly as a function of the student's continuous-valued latent abilities $\vect \Theta$ (e.g. add, subtract, multiply, divide) and parameters associated with the item, such as difficulty. Given a group of students' binary responses (correct/incorrect) to an assessment, parameter estimation techniques are used to infer student ability $\vect \Theta$ to better evaluate student performance. But as the number of students, items, and dimension of $\vect \Theta$ increases, traditional parameter estimation methods which rely on numerical integration or MCMC techniques become infeasible. 

In this thesis, a novel modification to a variational autoencoder (VAE), an unsupervised learning method, is presented which incorporates multiple pieces of domain knowledge into the VAE architecture. Specifically, an expert-annotated $Q$-matrix detailing the association between items and abilities is used to constrain the trainable weights in the VAE decoder. This directly links the generative posterior distribution of a VAE to IRT models, and allows interpretation of trainable weights as item parameters and a hidden neural layer as student ability estimates. The use of a neural network in this application allows for efficient estimation of paramaters for high-dimensional data.

An additional alteration to the VAE encoder allows for modeling of correlated, rather than independent, latent abilities. This architecture is also of interest to areas outside of education where the probability distribution of the VAE latent code is known. The proposed method, titled ML2P-VAE, acheives accuracy similar to traditional parameter estimation methods on smaller datasets, and provides quality estimates on educational assessments with a large number of students, items, and latent abilities where traditional methods struggle.

Modifications used in ML2P-VAE are extended to integrate IRT into deep knowledge tracing models, which use time-dependent neural networks such as LSTM and Transformers. In an online environment where many items are available, the task of knowledge tracing is to track student's learning dynamically as they progress through an assessment. This task has received more attention recently due to the emergence of online learning and AI tutoring systems, where following student's knowledge acquisition can help facilitate automated curriculum in real time. The inclusion of IRT in the knowledge tracing framework presents a trade-off between explainability and accuracy, while remaining competitive with state-of-the-art methods.


